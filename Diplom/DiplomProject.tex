\documentclass[a4paper,12pt]{article} %размер бумаги устанавливаем А4, шрифт 12пунктов
\usepackage[T2A]{fontenc}
\usepackage[utf8]{inputenc}	%кодировка
\usepackage[english,russian]{babel}%используем русский и английский языки с переносами
\usepackage{amssymb,amsfonts,amsmath,cite,enumerate,float,indentfirst} %пакеты расширений
\usepackage{graphicx} %вставка графики
%\graphicspath{{images/}}%путь к рисункам

\makeatletter
\renewcommand{\@biblabel}[1]{#1.} % Заменяем библиографию с квадратных скобок на точку:
\makeatother


%\usepackage{styles/main} 

\usepackage{geometry} % Меняем поля страницы
\geometry{left=2cm}% левое поле
\geometry{right=1.5cm}% правое поле
\geometry{top=1cm}% верхнее поле
\geometry{bottom=2cm}% нижнее поле

\renewcommand{\theenumi}{\arabic{enumi}}% Меняем везде перечисления на цифра.цифра
\renewcommand{\labelenumi}{\arabic{enumi}}% Меняем везде перечисления на цифра.цифра
\renewcommand{\theenumii}{.\arabic{enumii}}% Меняем везде перечисления на цифра.цифра
\renewcommand{\labelenumii}{\arabic{enumi}.\arabic{enumii}.}% Меняем везде перечисления на цифра.цифра
\renewcommand{\theenumiii}{.\arabic{enumiii}}% Меняем везде перечисления на цифра.цифра
\renewcommand{\labelenumiii}{\arabic{enumi}.\arabic{enumii}.\arabic{enumiii}.}% Меняем везде перечисления на цифра.цифра

\makeatletter
\renewcommand{\l@section}{\@dottedtocline{1}{0em}{2.1em}}
\renewcommand{\l@subsection}{\@dottedtocline{1}{0em}{2.1em}}
\renewcommand{\l@subsubsection}{\@dottedtocline{1}{0em}{0em}}
\makeatother



\begin{document}
	
\fontsize{14}{16pt}\selectfont	

\begin{titlepage}
\newpage

\begin{center}
\LARGE Московский государственный технический \\
\LARGE университет имени Н.Э.Баумана \\
\end{center}


\begin{center}
\large Факультет "Радиоэлектроника и лазерная техника" \\
\large Кафедра "Технологии приборостроения"
\end{center}

\begin{figure}[h]
	\centering
	\includegraphics[width = 0.4\linewidth]{logo-bmstu-big.png}
	 
\end{figure}

\vspace{2em}
\begin{center}
	\LARGE ВКР \\
	\LARGE на тему\\
	\vspace{2em}
	\Huge  Оптимизация процесса термического окисления кремниевых подложек \\
\end{center}









\vspace{3em}



\newbox{\lbox}
\savebox{\lbox}{\hbox{Шашурин В.Д.}}
\newlength{\maxl}
\setlength{\maxl}{\wd\lbox}
\hfill\parbox{11cm}{
\hspace*{5cm}\hspace*{-5cm}\large Студент:\hfill\hbox to\maxl{Власов Е.Ю.\hfill}\\
\hspace*{5cm}\hspace*{-5cm}\large Руководитель:\hfill\hbox to\maxl{Шашурин В.Д.}\\
\\
\hspace*{5cm}\hspace*{-5cm}Группа:\hfill\hbox to\maxl{РЛ6-82}\\
}


\vspace{\fill}

\begin{center}
\large Москва \\ \large 2018
\end{center}

\end{titlepage}% это титульный лист
\tableofcontents
\newpage
\section{Введение}

Объектом исследования данной работы являются \\


Предметом исследования данной работы являются \\


\subsection*{Цель работы}

- Получение равномерного слоя оксида кремния на поверхности подложки в результате термического окисления при различных температурах

\subsection*{Задачи работы}






\pagebreak

%\input{RefProject-Description}% это описание
%\input{RefProject-Algoritm}% это описание алгоритмов
%\input{RefProject-Finish}% заключение
%\input{RefProject-App}% приложение
%\newpage
%\tableofcontents % это оглавление, которое генерируется автоматически
\end{document}